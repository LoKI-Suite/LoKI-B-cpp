\documentclass{article}

\usepackage{hyperref}
\usepackage{amsmath}
\usepackage{bm}
% the next block is the modern way of achieving 'a4wide'
\usepackage{geometry}
\usepackage{layout}
\geometry{
  includeheadfoot,
  margin=2.54cm
}

% This includes the listings.sty style file and does some global configuration.
% The options frame= and captionpos= should _NOT_ be overridden in your .tex
% files in order to guarantee a consistent look of LoKI documents.
% The option language=C++ is present here because we have C++ code in most
% cases. It can be overridden in the lst-options list to whatever code you 
% are listing, of course.
% The syntax highlighting settings are subject to debate, IMHO
% the defaults were a bit too wild, hence the settings here.
%   --- Jan van Dijk, March 2001
%

\usepackage{listings}
\usepackage{upquote}
\usepackage{xcolor}
\lstset{
	frame=LBtr,
	captionpos=b,
	language=C++,
%	basicstyle=\small\itshape,
	basicstyle=\small\tt,
	keywordstyle=\bfseries,
%	commentstyle=\color{blue},
%	commentstyle=\tt,
	stringstyle=\slshape,
	basewidth={0.58em,0.47em},
	flexiblecolumns
}

% Use the command \SRC for inline code fragments. It uses the same style
% attributes as the rest of the 'listing' style, but overrides the frame option
% the @ sign is arbitrary, but should not occur in the code.
% In case we are being processed by Latex2Html we select a typewriter font
% and process the argument as is.
\newcommand{\SRC}[1]{{\lstinline[frame=]@#1@}}

\lstdefinestyle{BashInputStyle}{
	language=bash,
	basicstyle=\small\sffamily,
	keywordstyle=\ttfamily,
	frame=,
	backgroundcolor=\color{gray!10},
	columns=fullflexible,
	linewidth=0.9\linewidth,
	xleftmargin=0.01\linewidth
}
\newcommand{\bashSRC}[1]{{\lstinline[frame=,language=bash]^#1^}}

\newcommand\bashline{\lstinline[style=BashInputStyle]}

\lstnewenvironment{bashlist}%
	{ \lstset{style=BashInputStyle} }%
	{}

\newcommand{\etal}{{\it et al\/}\ }
\renewcommand{\vec}[1]{{\ensuremath{{\bm{#1}}}}}
\newcommand{\ten}[1]{{\ensuremath{{\mathbf{#1}}}}}
\newcommand{\mat}[1]{{\ensuremath{{\mathbf{#1}}}}}

\newcommand{\DERIV}[2]{\frac{\mathrm{d} {#1}\hfill}{\mathrm{d} {#2}\hfill}}
\newcommand{\DDERIV}[2]{\frac{\mathrm{d}^2 {#1}\hfill}{\mathrm{d} {#2}^2}\hfill}
\newcommand{\PDEV}[2]{\frac{\partial {#1}\hfill}{\partial {#2}\hfill}}
\newcommand{\PPDEV}[2]{\frac{\partial^2 {#1}\hfill}{\partial {#2}^2\hfill}}
\newcommand{\dd}[1]{\ensuremath{\mathrm{d}}{{#1}}}

\newcommand{\spec}[1]{{\ensuremath{\mathrm{#1}}}}


\bibliographystyle{unsrt}

\title{Notes on the C++ Implementation of LoKI-B}
\author{Jan van Dijk, Daan Boer and Wouter Graef}

\begin{document}

\maketitle

\abstract{
This document contains notes about the C++ version of LoKI-B.
It provides additional details of definitions and algorithms that are not
explicitly written in the LoKI-B paper \cite{Tejero2019} or the User Manual
of the initial public release of LoKI-B \cite{Manual_1_0_0}. It also
discusses features of the C++ version that are not available yet in the MATLAB
version, such as support for JSON input files and the Web deployment of LoKI-B
and discusses and motivates things that have been done different in the C++
version for various reasons.
}

\tableofcontents

\section{Introduction}

\subsection{Definitions}

There are multiple variables that describe the densities (absolute and relative)
of the gases and states.
\begin{itemize}
  \item $N$
  \item $N_k$
  \item $\chi_k$
  \item $\xi_{k_i}$
  \item $\delta_{k_i}$
\end{itemize}

In the Loki-B paper \cite{Tejero2019}, in the text between equations 1 and 2,
$\xi_{k_i}$ is defined as the density of a state $k_i$ divided by that of the
sum of all densities of states of gas $k$. On the other hand, in the LoKI-B
user manual (version 1.0.0) \cite{Manual_1_0_0} below equation 2 it is defined
as the density of that state divided by its parent density. TODO: what is in
MATLAB? Let's adopt that and state that clearly here. SOme names in the code
must perhaps be changed: \SRC{density} in the code corresponds to the `reduced
density' in the user guide, it seems. Call that \SRC{state_density}?

Gamma has been introduced. This constant also appears in the Bolsig+ paper
\cite{Hagelaar2005}, where it introduced just below equation 6, and is defined
as
\begin{equation}
  \gamma = \sqrt{2e/m_e}.
\end{equation}
The symbol is available as \SRC{SI::gamma} (relative to the \SRC{loki}
namespace). Note that in LoKI-B documentation the symbol $\gamma_k$ is
defined as the mass ration of the electron and a species $k$
\cite[below eqn. 9d]{Manual_1_0_0},
\begin{equation}
  \gamma_k = m_e/m_k,
\end{equation}
but the presence of the subscript $k$ results in a clear distinction.

\section{Matrices \& Vectors}

Eigen, the various Hessenberg-solvers.

\subsection{EEDF Normalization Strategies}

The EEDF is ormalized by the condition that $\int\limits_0^\infty \sqrt{u}f(u)\dd{u}=1$,
which can be achieved in two ways:
\begin{enumerate}
  \item Make the equation $\sum_c \sqrt{u_c} f(u_c) \Delta u=1$ part of the
    linear system of equations. This can be achieved for example by filling the
    columns $s$ of the first row of the system matrix with values
    $\sqrt{u[c]}\Delta u$, and setting the right hand side to 1.
  \item Impose the condition $f(u_0)=1$, then solve the equation and do the
    scaling of $f(u)$ that is dictated by the normalization condition afterwards.
\end{enumerate}
It appears that at present both are done (in both MATLAB and C++ versions). The
advantage of (only) using the second mathod is that no unnecessary non-zero
entries are introduced in the system matrix. This is beneficial when a sparse
matrix solver is adopted. Especially when a band matrix format is used, it
prevents all upper diagonals to be allocated, in most cases only containing
a single non-zero element due to the normalization condition.

\section{Build System}

\subsection{cmake}

\subsection{autotools}

\subsection{Cross Compilation}

\subsection{Automated Testing}

Unit tests, checking results with reference data, comparing the MATLAB and C++
versions, comparisons with Bolsig+ output.

\bibliography{../../doc/reflist}

\end{document}
