\documentclass{article}

\usepackage{framed}
\usepackage{hyperref}
\usepackage{amsmath}
\usepackage{bm}
% the next block is the modern way of achieving 'a4wide'
\usepackage{geometry}
\usepackage{layout}
\geometry{
  includeheadfoot,
  margin=2.54cm
}
\usepackage{xfrac}
\newcommand{\frachalf}{\ensuremath{\sfrac{1}{2}}}
\newcommand{\phalf}[1]{\ensuremath{{#1}+\frachalf}}
\newcommand{\mhalf}[1]{\ensuremath{{#1}-\frachalf}}

\newcommand{\MEAN}[1]{\ensuremath{\langle{{#1}}\rangle}}

% This includes the listings.sty style file and does some global configuration.
% The options frame= and captionpos= should _NOT_ be overridden in your .tex
% files in order to guarantee a consistent look of LoKI documents.
% The option language=C++ is present here because we have C++ code in most
% cases. It can be overridden in the lst-options list to whatever code you 
% are listing, of course.
% The syntax highlighting settings are subject to debate, IMHO
% the defaults were a bit too wild, hence the settings here.
%   --- Jan van Dijk, March 2001
%

\usepackage{listings}
\usepackage{upquote}
\usepackage{xcolor}
\lstset{
	frame=LBtr,
	captionpos=b,
	language=C++,
%	basicstyle=\small\itshape,
	basicstyle=\small\tt,
	keywordstyle=\bfseries,
%	commentstyle=\color{blue},
%	commentstyle=\tt,
	stringstyle=\slshape,
	basewidth={0.58em,0.47em},
	flexiblecolumns
}

% Use the command \SRC for inline code fragments. It uses the same style
% attributes as the rest of the 'listing' style, but overrides the frame option
% the @ sign is arbitrary, but should not occur in the code.
% In case we are being processed by Latex2Html we select a typewriter font
% and process the argument as is.
\newcommand{\SRC}[1]{{\lstinline[frame=]@#1@}}

\lstdefinestyle{BashInputStyle}{
	language=bash,
	basicstyle=\small\sffamily,
	keywordstyle=\ttfamily,
	frame=,
	backgroundcolor=\color{gray!10},
	columns=fullflexible,
	linewidth=0.9\linewidth,
	xleftmargin=0.01\linewidth
}
\newcommand{\bashSRC}[1]{{\lstinline[frame=,language=bash]^#1^}}

\newcommand\bashline{\lstinline[style=BashInputStyle]}

\lstnewenvironment{bashlist}%
	{ \lstset{style=BashInputStyle} }%
	{}

\newcommand{\etal}{{\it et al\/}\ }
\renewcommand{\vec}[1]{{\ensuremath{{\bm{#1}}}}}
\newcommand{\ten}[1]{{\ensuremath{{\mathbf{#1}}}}}
\newcommand{\mat}[1]{{\ensuremath{{\mathbf{#1}}}}}

\newcommand{\DERIV}[2]{\frac{\mathrm{d} {#1}\hfill}{\mathrm{d} {#2}\hfill}}
\newcommand{\DDERIV}[2]{\frac{\mathrm{d}^2 {#1}\hfill}{\mathrm{d} {#2}^2}\hfill}
\newcommand{\PDEV}[2]{\frac{\partial {#1}\hfill}{\partial {#2}\hfill}}
\newcommand{\PPDEV}[2]{\frac{\partial^2 {#1}\hfill}{\partial {#2}^2\hfill}}
\newcommand{\dd}[1]{\ensuremath{\mathrm{d}}{{#1}}}

\newcommand{\spec}[1]{{\ensuremath{\mathrm{#1}}}}


\bibliographystyle{unsrt}

\title{Notes on the C++ Implementation of LoKI-B}
\author{Jan van Dijk, Daan Boer and Wouter Graef}

\begin{document}

\maketitle

\abstract{
This document contains notes about the C++ version of LoKI-B.
It provides additional details of definitions and algorithms that are not
explicitly written in the LoKI-B paper \cite{Tejero2019} or the User Manual
of the initial public release of LoKI-B \cite{Manual_1_0_0}. It also
discusses features of the C++ version that are not available yet in the MATLAB
version, such as support for JSON input files and the Web deployment of LoKI-B
and discusses and motivates things that have been done different in the C++
version for various reasons.
}

\tableofcontents

\section{Definitions}

\subsection{Basic Quantities}

There are multiple variables that describe the densities (absolute and relative)
of the gases and states.
\begin{itemize}
  \item $N$
  \item $N_k$
  \item Gas fraction:
  \begin{equation}
    \chi_k = N_k/N.
    \label{eq:chi_k}
  \end{equation}
  \item $\xi_{k_i}$
  \item $\delta_{k_i}$
\end{itemize}

\begin{itemize}
  \item In the Loki-B paper \cite{Tejero2019}, in the text between equations 1
  and 2, $\xi_{k_i}$ is defined as the density of a state $k_i$ divided by that
  of the sum of all densities of states of gas $k$. On the other hand, in the
  LoKI-B user manual (version 1.0.0) \cite{Manual_1_0_0} below equation 2 it
  is defined as the density of that state divided by its parent density.
  TODO: what is in MATLAB? Let's adopt that and state that clearly here.
  \item The name \SRC{density} that appeared in the code corresponded to the
  `reduced density' $\delta_{k}$ in the user guide. In the C++ code this is
  now provided by member \SRC{delta()} of the state class.
\end{itemize}

\subsection{The gamma constant}

Gamma has been introduced. This constant also appears in the BOLSIG+ paper
\cite{Hagelaar2005}, where it introduced just below equation 6, and is defined
as
\begin{equation}
  \gamma = \sqrt{2e/m_e}.
  \label{eq:gamma}
\end{equation}
The symbol is available as \SRC{SI::gamma} (relative to the \SRC{loki}
namespace). Note that this appears {\em many} times in the code, and
this change has resulted in much better readability.

Note that in LoKI-B documentation the symbol $\gamma_k$ is
defined as the mass ration of the electron and a species $k$
\cite[below eqn. 9d]{Manual_1_0_0},
\begin{equation}
  \gamma_k = m_e/m_k,
\end{equation}
but the presence of the subscript $k$ avoids any confusion with the
constant $\gamma$, which in addition has the \SRC{SI::} prefix.

\subsection{The functions $G_x(y)$ and $g_x(u)$}

Many terms on the left-hand side of the 2-term approximation of the Boltzmann
equation are written in terms of energy-derivatives of functions $G_x(u)$, where
$x$ denotes a particular process type.
\begin{itemize}
  \item Expressions for $G_x(u)$ are given in equations 6b-c of the LoKI-B paper
        \cite{Tejero2019} and in equations 9b-c of the manual \cite{Manual_1_0_0};
        those are identical.
\end{itemize}
Of course, the Boltzmann equation does not change if each term is divided by the
same non-zero constant value. That does, however, change the meaning/interpretation
of the individual terms and has resulted in conflicting definitions for
auxiary functions $g_x(u)$.
\begin{itemize}
  \item In the paper \cite[eqn. 15d]{Tejero2019}, $g_x(u)$ and $c_u$ are defined
        by the relation
        \[
          G_x(u) = g_x(u)\left(f(u)+c_x\DERIV{f(u)}{u}\right),
        \]
        of which the energy-derivative is stated in the text. As an example,
        for the elastic operator this results in \cite[eqn. 6b]{Tejero2019},
        \[
           g_{el}(u)=-\sum\limits_k 2\gamma_k\nu^{el}_{k,c}(u)u^{3/2}
           = -N\sqrt{2e/m_e}\;2u^2\sum\limits_k \delta_k\gamma_k\sigma^{el}_{k,c}(u)
           := -N\gamma\;2u^2\sigma^{el}_u(u).
        \]
        (TODO: add note about the 'eneragy averaged cross section here, including $2m_e/m_k$.)
        This matches the {\em comments} in the MATLAB code, which announce
\begin{lstlisting}
  g_c(u) = -N*sqrt(2*e/me)*2*u^2*sigmaC(u);
\end{lstlisting}
  \item In the user manual \cite[eqn. 16d]{Tejero2019}, $g_x(u)$ is defined
        relative to $G_x(u)/(N\gamma)$, that is:
        \begin{equation}
          \frac{1}{N\gamma}G_x(u) = g_x(u)\left(f(u)+c_x\DERIV{f(u)}{u}\right).
          \label{eq:G(g)}
        \end{equation}
        That would instead result in:
\begin{lstlisting}
  g_c(u) = -2*u^2*sigmaC(u);
\end{lstlisting}
  \item In the actual implementation we find the latter definition, but {\em without
        the minus sign}: the implementation of the elastic operator actually does
\begin{lstlisting}
  boltzmann.g_c = 2*boltzmann.energyGrid.node.^2.*boltzmann.elasticCrossSection;
\end{lstlisting}
        So the function $g_x(u)$ that we actually find in the code corresponds to the definition
        \[
          \frac{1}{N\gamma}G_x(u) = -g_x(u)\left(f(u)+c_x\DERIV{f(u)}{u}\right).
        \]
  \item Since the expressions for the discretization coefficients in terms of
        $g_x(u)$ are the same in the code as in the manual, the effect is that
        the matrices $M^x$ that are set up such that
        \[
           [M^x_{ij}][f_j] \doteq -\frac{1}{N\gamma}\DERIV{G_x(u)}{u}.
        \]
        (Also see section \ref{sec:discretization}.)
  \item Note that the elements of $g_x(u)$ are also used to calculated the
        elastic terms on the power balance, see \cite[eq. 24c]{Tejero2019} and
        \cite[eq. 38c]{Manual_1_0_0}.\footnote{
          In the code, a different expression is used, that appears to be
          based on partial integration of the expressions in the paper and
          manual. This must be checked.}
        When the meaning of $g_x(u)$ is changed, also those expressions must
        be updated accordingly.
  \item In order to compare expressions with those in the BOLSIG+ paper
        \cite{Hagelaar2005}, it is important to realize that in that paper the
        terms of the equation correspond to the LoKI-B expressions
        \[
          \frac{1}{N}\DERIV{G_x(u)}{u},
        \]
        in other words: there is no division by $\gamma$. As an example, consider
        \[
          \frac{1}{N}G_{el}(u)
          = -\gamma\;2u^2\sigma^{el}_u(u)\left[f(u) + \frac{k_BT}{e}\DERIV{f(u)}{u}\right].
        \]
        This corresponds to the elastic part (involving $\sigma_\epsilon$) of equations
        40 and 41 of \cite{Hagelaar2005}.
        TODO: note the definition of $\sigma_\epsilon$ in equation (42) in
        \cite{Hagelaar2005}. The correspondence with the expression in LoKI-B should
        be made explicit, the mass-ratios emphasized.
\end{itemize}

\subsection{The CAR Implementation}

The CAR term in the EBE is described by equation 6c of \cite{Tejero2019}.
It can be written in the form \eqref{eq:G(g)}, with
\begin{align}
  N\gamma g_{CAR} &= -\sum\limits_k 4B_k\nu_{0,k}\sqrt{u}, \\
  c_{CAR} &= k_BT_e/e.
\end{align}
The frequency of the CAR cross section for a gas $k$ is presented just below
equation 6d of \cite{Tejero2019}. Using the definition of the gas fraction
$\chi_k$ from equation \eqref{eq:gamma} this becomes
\begin{equation}
  \nu_{0,k} = N_k\sqrt{2eu/m_e}\sigma_{0,k} = N\gamma\sqrt{u}\chi_k\sigma_{0,k}
\end{equation}
and substitution in the expression for $g_{CAR}$ yields
\begin{equation}
  g_{CAR} = -4u\sum\limits_k \chi_k\sigma_{0,k}B_k.
  \label{eq:g_CAR}
\end{equation}
\begin{framed}
  As discussed elsewhere, in the code we find $g_{CAR}$ defined without
  the minus sign --- the same is true for other $g_x(u)$.
\end{framed}

The CAR cross section for a gas $k$ can be found in the LoKI-B paper \cite{Tejero2019}
in the text below equation 6d, which states
\begin{equation}
  \sigma_{0,k} = \frac{8\pi}{15}Q_{k,au}^2a_0^2,
  \label{eq:sigma_CAR}
\end{equation}
where $a_0$ is the Bohr radius and $Q_{k,au}$ is the quadrupole moment of that
gas {\em in atomic units} ($ea_0^2$).
This expression is also found in the user manual \cite{Manual_1_0_0} (below
equation 9d) and in the paper of Ridenti \etal \cite{Ridenti2015} (below
equation 8b). Those sources refer to the paper of Gerjuoy and Stein
\cite{Gerjuoy1955}, where this term appears as a factor in equation 20.
To emphasize that $Q_{k,au}$ is a dimensionless number (the numerical value
that arises when atomic units are used), we have added the suffix $au$ in equation
\eqref{eq:sigma_CAR}; this does not appear in the sources cited above.

Substitution of this expression in \eqref{eq:g_CAR} gives
\begin{equation}
  g_{CAR} = -4u\frac{8\pi}{15}a_0^2\sum\limits_k \chi_k Q_{k,au}^2B_k.
\end{equation}

\begin{framed}
Note (December 14 2020) ---
In version 1.0.0 of the LoKI-B code (MATLAB), $\sigma_{0,k} = \frac{8\pi}{15}Q_{k,au}a_0^2$,
was used (without the square on $Q_{k,au}$). Since $Q_{k,au}=Q_k/(ea_0^2)$, where
$Q_k$ is the value in SI units, that would result in $\sigma_{0,k}=(8\pi/15)Q_k/e$
and in
\begin{equation}
  \sum\limits_k\chi_k\sigma_{0,k}B_k = \frac{8\pi}{15}\frac{1}{e}\sum\limits_k \chi_k Q_kB_k,
\end{equation}
and this is what was found in the code in \SRC{Boltzmann.m} line 423,
in function \SRC{evaluateCAROperator}:
\begin{lstlisting}[language=matlab]
sigma0B = 0;
for gasName = boltzmann.CARgases
  gasID = Gas.find(gasName, boltzmann.gasArray);
  gas = boltzmann.gasArray(gasID);
  sigma0B = sigma0B + gas.fraction*gas.electricQuadrupoleMoment*gas.rotationalConstant;
end
sigma0B = 8.0*pi*sigma0B/(15.0*Constant.electronCharge);
\end{lstlisting}
(Note that in the MATLAB code indeed \SRC{electricQuadrupoleMoment} is expressed
in SI units, see file \SRC{Input/Databases/quadrupoleMoment.txt}.)

Since $Q_{k,au}$ is usually of order unity, it is no surprise
that this could go unnoticed. The implementation has been corrected to
\begin{lstlisting}[language=matlab]
a02 = Constant.bohrRadius*Constant.bohrRadius:
sigma0B = 0;
for gasName = boltzmann.CARgases
  gasID = Gas.find(gasName, boltzmann.gasArray);
  gas = boltzmann.gasArray(gasID);
  Qau = gas.electricQuadrupoleMoment/(Constant.electronCharge*a02);
  sigma0B = sigma0B + gas.fraction*Qau*Qau*gas.rotationalConstant;
end
sigma0B = (8.0*pi*a02/15.0)*sigma0B;
\end{lstlisting}
Previous versions of the C++ implementation had the same problem, it
has been fixed along the same lines.

\end{framed}


\section{Discretization}
\label{sec:discretization}

\subsection{Final form of the equation}
In \cite[eqn. 3a]{Tejero2019} the final form of the two-term approximation of the
Boltzmann equation with a temporal growth term is written as
\begin{equation}
  \frac{1}{N\gamma}\MEAN{\nu_{\mbox{eff}}}uf(u) + \frac{1}{N\gamma}\DERIV{G(u)}{u} = S(u).
\end{equation}
The equation with the spatial graowth model is similar, only the first term takes
a different form. The equation is subject to the normalization condition
\begin{equation}
  \int\limits_0^\infty \sqrt{u}f(u)du=1.
  \label{eq:f-norm}
\end{equation}
Bringing all terms of the EBE to the right-hand side of the equation yields
\begin{equation}
  -\frac{1}{N\gamma}\MEAN{\nu_{\mbox{eff}}}uf(u) - \frac{1}{N\gamma}\DERIV{G(u)}{u} + S(u) = 0.
  \label{eq:EBE-temp}
\end{equation}
In LoKI-B, this particular form of the equation is cast in matrix-vector form,
\begin{equation}
  \sum\limits_{j=0}^{N_c-1}C_{ij}f_j = 0.
  \label{eq:Cf=0}
\end{equation}
In this equation, the system matrix $C_{ij}=\sum_x C_{x,ij}$, where each
matrix $C_{x,ij}$ corresponds to one of a terms, identified by $x$, in
equation \eqref{eq:EBE-temp}. Then, for example
\[
   \sum\limits_{j=0}^{N_c-1}C_{el,ij}f_j
   \doteq
   -\frac{1}{N\gamma}\left.\DERIV{G_{el}(u)}{u}\right|_{u_i}.
\]
Note that a constant multiplication factor could still be applied to the matrix
$C_{ij}$ in equation \eqref{eq:Cf=0} without changing the solution, but that
would change the physical meaning of an expression like $\sum_jC_{el,ij}f_j$.
Note that for some terms the coefficients depend on $f$, making this a
{\em non-linear system}.

TODO: verify that the above is indeed what is done by LoKI-B (and be very precise).

Since the system is homogeneous, the solution is determined up to a constant
multiplication factor. This follows from the normalization condition
\eqref{eq:f-norm}, which can be approximated as
\begin{equation}
  \sum\limits_{j=0}^{N_c-1} \sqrt{u_j} f_j \Delta u = 1.
  \label{eq:f-norm-disc}
\end{equation}
This condition can be imposed ion the system in multiple ways:
\begin{itemize}
  \item Replace equation \eqref{eq:Cf=0} with the discrete form of the
    normalization condition as given by \eqref{eq:f-norm-disc}.
  \item Replace this equation with the equation $f_0=1$ (or another fixed
    number), solve the system and scale the solution afterwards.
\end{itemize}
It appears that at present both are done (in both MATLAB and C++ versions). The
advantage of (only) using the second mathod is that no unnecessary non-zero
entries are introduced in the system matrix. This is beneficial when a sparse
matrix solver is adopted. Especially when a band matrix format is used, it
prevents all upper diagonals to be allocated, in most cases only containing
a single non-zero element due to the normalization condition.

\subsection{Discretization of the Energy Transport Terms}

The LoKI-B paper \cite{Tejero2019} discusses the numerical recipes in
section 3.4. There it is stated that the finite-difference method of
Rockwood \cite[appendix A]{Rockwood1973} is used. The relation between Rockwood's
discretization strategy and the one adopted in LoKI-B is not immediately
clear, since Rockwoord provides explicit expression for the two terms
he considers (field and elastic terms), not the general recipe.

Rockwood approximates $\dd{G(u)}/\dd{u}$ in the cell points $k$ as
\begin{equation}
  \left.\DERIV{G_x(u)}{u}\right|_k
  = \frac{G_x(u_{\phalf{k}}) - G_x(u_{\mhalf{k}})}{\Delta u}.
\end{equation}
It is important to realize that this amounts to a finite-{\em volume} method,
since the right-hand side can also be written as
\[
  \frac{1}{\Delta u}\left[G_x(u_{\phalf{k}}) - G_x(u_{\mhalf{k}})\right]
  =
  \frac{1}{\Delta u}\int\limits_{u_{\mhalf{k}}}^{u_{\phalf{k}}}\DERIV{G_x(u)}{u}\dd{u},
\]
which makes clear that the approximation amounts to replacing the value of
$\dd G_x/\dd u$ with its mean value in the interval
$[u_{\mhalf{k}},u_{\phalf{k}}]$. Except for the factor $1/\Delta u$ this is,
of course, the volume integral of $\dd G_x(u)/\dd u$ over that interval.
All this can be recognized as applying Gau\ss' theorem to an integral over
a one-dimensional energy volume, with flux $G(u)$ and flux-divergence
$\dd G(u)/\dd u$.

This analysis makes clear that the emphasis can be placed on a proper discretization
of the flux $G(u)$ on the boundaries $u_\mhalf{k}$ and $u_\phalf{k}$ of the cell
around $u_k$. LoKI-B follows Rockwood in adopting a central difference scheme for
the approximation of equation \eqref{eq:G(g)}. Using the definition of $G_x(u)$
that appears in the user manual, see equation \ref{eq:G(g)} of the present text,
we get
\begin{align}
  \frac{1}{N\gamma}G_x(u_{\phalf{k}})
    &\doteq g_x(u_{\phalf{k}})\left(\frac{f(u_{k+1})+f(u_k)}{2}+c_x\frac{f(u_{k+1})-f(u_k)}{\Delta u}\right) \nonumber \\
    &\doteq
      -g_x(u_{\phalf{k}})\left(\frac{c_x}{\Delta u}-\frac{1}{2}\right)f(u_k)
      +
      g_x(u_{\phalf{k}})\left(\frac{c_x}{\Delta u}+\frac{1}{2}\right)f(u_{k+1})
  \\
  \frac{1}{N\gamma}G_x(u_{\mhalf{k}})
    &\doteq g_x(u_{\mhalf{k}})\left(\frac{f(u_{k})+f(u_{k-1})}{2}+c_x\frac{f(u_{k})-f(u_{k-1})}{\Delta u}\right) \nonumber \\
    &\doteq
      g_x(u_{\mhalf{k}})\left(\frac{c_x}{\Delta u}+\frac{1}{2}\right)f(u_k)
      -
      g_x(u_{\mhalf{k}})\left(\frac{c_x}{\Delta u}-\frac{1}{2}\right)f(u_{k-1})
\end{align}
Which results in the following discrete result,
\begin{align}
  \frac{1}{N\gamma}\DERIV{G_x(u)}{u}
  =& \frac{1}{N\gamma}\frac{G_x(u_{\phalf{k}}) - G_x(u_{\mhalf{k}})}{\Delta u} \nonumber \\
       =& +\left[\frac{g_x(u_{\mhalf{k}})}{\Delta u}\left(\frac{c_x}{\Delta u}-\frac{1}{2}\right)\right]f(u_{k-1}) \nonumber \\
        & -\left[
              \frac{g_x(u_{\mhalf{k}})}{\Delta u}\left(\frac{c_x}{\Delta u}+\frac{1}{2}\right)
            + \frac{g_x(u_{\phalf{k}})}{\Delta u}\left(\frac{c_x}{\Delta u}-\frac{1}{2}\right)
           \right] f(u_k) \nonumber \\
        & +\left[\frac{g_x(u_{\phalf{k}})}{\Delta u}\left(\frac{c_x}{\Delta u}+\frac{1}{2}\right)\right]f(u_{k+1}) \nonumber \\
       =& M(k,k-1)f(u_{k-1}) + M(k,k)f(u_k) + M(k,k+1)f(u_{k+1}).
\end{align}
NOTE: A relation between the central coefficient $M(k,k)$ and the neighbour
coefficients can be derived as follows:
\begin{align}
  -M(k,k)
    &=
    \frac{g_x(u_{\mhalf{k}})}{\Delta u}\left(\frac{c_x}{\Delta u}+\frac{1}{2}\right)
    +
    \frac{g_x(u_{\phalf{k}})}{\Delta u}\left(\frac{c_x}{\Delta u}-\frac{1}{2}\right) \nonumber \\
    &=
    \frac{g_x(u_{\mhalf{k}})}{\Delta u}\left(\frac{c_x}{\Delta u}-\frac{1}{2}\right)
    +
    \frac{g_x(u_{\phalf{k}})}{\Delta u}\left(\frac{c_x}{\Delta u}+\frac{1}{2}\right)
    + \frac{g_x(u_{\mhalf{k}})}{\Delta u} - \frac{g_x(u_{\phalf{k}})}{\Delta u} \nonumber \\
    &=
    M(k,k-1)
    +
    M(k,k+1)
    + \frac{g_x(u_{\mhalf{k}}) - g_x(u_{\phalf{k}})}{\Delta u}.
\end{align}
this can be used later on to do some stability analysis on the scheme. In brief:
consider a situation where this is the only term in the Boltzmann equation, and
assume that $g_x(u_{\mhalf{k}})=g_x(u_{\phalf{k}})$ for some cell $k$. Then we can
derive
\begin{align}
  &M(k,k-1)f(u_{k-1}) + M(k,k)f(u_k) + M(k,k+1)f(u_{k+1}) = 0 \nonumber \\
  \implies & M(k,k)f(u_k) = - M(k,k+1)f(u_{k+1}) - M(k,k-1)f(u_{k-1}) \nonumber \\
  \implies & (M(k,k-1)+M(k,k+1))f(u_k) = + M(k,k+1)f(u_{k+1}) + M(k,k-1)f(u_{k-1}) \nonumber \\
  \implies & f(u_k) = (1-\alpha)f(u_{k+1}) + \alpha f(u_{k-1}),
\end{align}
where $\alpha=M(k,k-1)/(M(k,k-1)+M(k,k+1))$. Only if the coefficients $M(k,k-1)$
and $M(k,k+1)$ have the same sign will this result in monotonic behaviour of the solution
$f(u)$. From this we can deduce the requirement that $||c_x/\Delta u||>1/2$, or
$||\Delta u/c_x||<2$. This number can be recognized as the {\em grid P\'eclet number}
of the problem. The condition can be eliminated by using another scheme than the
central difference scheme (upwind, Scharfetter-Gummel, ...).

Note that a true finite-difference scheme (not in LoKI-B) would have been obtained by
first employing the product rule for differentiation and grouping terms with the
same derivative of $f(u)$,
\begin{equation}
  \frac{1}{N\gamma}G_x(u)
    = \DERIV{g_x(u)}{u}f(u)
    + \left(g_x(u)+\DERIV{g_x(u)}{u}c_x\right)\DERIV{f(u)}{u}
    + g_x(u)c_x\DDERIV{f(u)}{u},
\end{equation}
then use $f'(u_k)\doteq(f(u_{k+1})-f(u_{k-1}))/(2\Delta u)$ and
$f''(u_k)\doteq(f(u_{k+1})+f(u_{k-1})-2f(u_k))/(\Delta u)^2$, etc.



\section{Matrices \& Vectors}

\subsection{Calculating the Maximum Relative Difference of Two Vectors}

During the development of the C++ code, a nasty difference between MATLAB
and C++ was discovered in the handling of NaN (not-a-number values) values,
that are produced by such operations as $0.0/0.0$, $\infty\cdot 0.0$ and
$\infty/\infty$, where $\infty$ indicates the platform's \SRC{inf} value.

When looking for the maximum element of an array, MATLAB skips NaN's and
returns the highes non-NaN element.
\begin{lstlisting}[language=matlab]
>> max(abs([0 0]./[1 0]))
ans =
     0
\end{lstlisting}
On the other hand, C++ with Eigen returns NaN. The last line of the code
below shows the calculation that was previously in the C++ version of LoKI-B,
it prints a NaN:
\begin{lstlisting}[language=matlab]
    #include <eigen3/Eigen/Dense>
    ...
    using Vector = Eigen::VectorXd;
    Vector ev1(2); ev1[0] = 0.; ev1[0] = 0.;
    Vector ev2(2); ev2[0] = 1.; ev2[0] = 0.;
    std::cout << (ev1.array()/ev2.array()).cwiseAbs().maxCoeff() << std::endl;
\end{lstlisting}
For this reason, a member \SRC{maxRelDiff} has been introduced. That avoids
the spurious NaN's by skipping over points for which both values (old and new)
are different. NOTE that this implements the relative {\em difference}, using
the averaged absolute values as reference, rather than the olf or new one.

\subsection{Linear Systems, Matrix Inversion}

Eigen, the various Hessenberg-solvers.

\subsection{The Smart Grid}

The smart grid functionality can be used to let the program adjust the
upper boundary of the energy grid such that the dynamic range of the
EEDF is within a user-specified specific range. The number of decades
that the EEDF drops is given by the expression
\begin{equation}
    N_d = \log_{10} f_{\mbox{max}} - \log_{10} f_{\mbox{min}}.
\end{equation}
LoKI-B assumes that the highest and lowest EEDF values are attained on
the first and last energy grid point, respectively. This is the expression
that is also implemented in the MATLAB version of LoKI-B. In case of
an underflow for high energy values, such that $f_{\mbox{min}}=0$, while
$f_{\mbox{max}}$ is finite, MATLAB produces the value \SRC{Inf}, since
\SRC{log10} is specified to return \SRC{-Inf} for zero-valued arguments\footnote{
See: \url{https://nl.mathworks.com/help/matlab/ref/log10.html}}.
In such case, a test like \SRC{Inf > N_max} will evaluate as \SRC{true}
and this tells LoKI-B to reduce the upper boundary of the energy grid.
Also the C/C++ function \SRC{log10} is specified to return \SRC{-Inf}
for zero-valued arguments.

It is tempting to combine the logarithms in this expression, wich would
result in
\begin{equation}
    N_d= \log_{10} \frac{f_{\mbox{max}}}{f_{\mbox{min}}}.
\end{equation}
It is important to realize that these expressions may give different
results in the case of underflow, when {\em signed} zeros are produced:
the IEE754 floating point standard recognizes \SRC{0.0} and the
signed zero \SRC{-0.0} as separate numbers, although they do compare
equal\footnote{See: \url{https://en.wikipedia.org/wiki/Signed_zero}}.
However, for a positive finite number $f$ we have \SRC{f/(+0.0)=+Inf}
and \SRC{f/(-0.0)=-Inf}, and while \SRC{log10(+Inf)=+Inf}, we have
\SRC{log10(-Inf)=NaN}. This means that for a positive $f_{\mbox{max}}$ we get
\begin{equation}
    \log_{10} f_{\mbox{max}} - \log_{10}(\pm 0.0)
  = \log_{10} f_{\mbox{max}} - (-\SRC{Inf}) = \SRC{+Inf}.
\end{equation}
whereas
\begin{equation}
  \log_{10} (f_{\mbox{max}}/(\pm 0.0)) = \log_{10}(\pm\SRC{Inf}),
\end{equation}
which gives \SRC{NaN} instead of \SRC{Inf} for a signed zero.

In the C++ code we have implemented the function \SRC{calcDecades(v1,v2)}
for this calculation. It has been implemented to return
\SRC{log10(abs(v1/v2)}. Taking the absolute value allows the calculation
to rely on only one calculation of a logithm, while avoiding the problem
that a \SRC{NaN} is produced for signed zeros. It also allows a comparison
of the size of scalars that are both negative or have different signs.
Perhaps most importantly, it centralizes this simply but tricky bit of
code, so it does not have to be repeated three times in the code.


\section{Build System}

\subsection{cmake}

\subsection{autotools}

\subsection{Cross Compilation}

\subsection{Automated Testing}

Unit tests, checking results with reference data, comparing the MATLAB and C++
versions, comparisons with BOLSIG+ output.

\bibliography{../../doc/reflist}

\end{document}
